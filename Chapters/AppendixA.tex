% Appendix A
% add following line for typesetting from subfiles
% !TeX root = ../uet_thesis.tex
\chapter{Matlab Implementation}
\label{AppendixA}
\lhead{Appendix A. \emph{Matlab Implementation}}


\section{Encoder Algorithm}
\begin{lstlisting}[style=Matlab-editor,basicstyle=\footnotesize]
function [out]=encode2ts(n,r,m)
% Encodes an integer to multi pulse position modulation(MPPM) array
% ENCODE2TS(n,r,IN_WORD) outputs a vector of 1's and 0's representing
% the input number 'IN_WORD' in 'n' bit codeword with 'r' 1's
% Most significat digit at hihgest index.
% n=8;r=4;m=9; encode2ts(n,r,m); > 1 1 0 0 1 1 0 0
% Input range for IN_WORD: 0 to nCr-1
% IN_WORD greater than nCr-1 results in array with all 1's
    out=zeros(1,n);
    while n>0
        if (n>r & r>=0) y=nchoosek(n-1,r);
        else y=0;
        end%end if
        if m>=y m=m-y;out(n)=1;r=r-1;
        else out(n)=0;
        end%end if
        n=n-1;
    end%while
end%function
\end{lstlisting}

\newpage
\section{Decoder Algorithm}



%\begin{verbatim}
\begin{lstlisting}[style=Matlab-editor,basicstyle=\footnotesize]
function out=decode2ts2(in_stream)
%in_stream=[1 1 1 0 1 0 1 1]  : out_num=14
%Most significant bit at highest array indices

r=1;					%counts the number of ones in input array (aka pulsed slots)
out_num=0; 			%calculate and store out_num
for n=1:length(in_stream) 	%Iterate from LSB to MSB in input array
    if (in_stream(n)==1)   	%If bit at this index is one then                       %out_number is incremented by Muqami
                            %Qeemat of the digit
    if (n > r) %Calculate Muqammi Qeemat from combination value for 
                   %choose(n,r). choose(n,r) is the possible number of
                   %combination if r objects are taken from total of n 
                   %objects. Here the function value is taken as zero if
                   %r is greater than or equal to n
                   %Muqammi Qeemat of a digit changes in two ways:(1)
                   %1. Position of the digit in the array [indexed by n]
                   %2. Position of the 1 in 1's in the array [indexed by r]
            out_num=out_num+nchoosek(n-1,r);
        end
        r=r+1; % Increment the number of 1s in array
    end
end
%\end{verbatim}
\end{lstlisting}

\newpage
\section{Calculate VR-MPPM Code Translation Matrix}
\begin{lstlisting}[style=Matlab-editor,basicstyle=\footnotesize]
%------------------------------------------------
%  getCodeMatrix(code_word_size,pulsed_slots)
%------------------------------------------------
function [nCodes code_matrix] = getCodeMatrix(codeword_size, pulsed_slots)

    nCodes = 2^(floor(log2(nchoosek(codeword_size, pulsed_slots))));
    code_matrix = [];
    % Extra codes discarded.
    for i = 1:nCodes
        code_matrix = [code_matrix; encode2ts(codeword_size, pulsed_slots, i)];
    end
    code_matrix;
end
\end{lstlisting}

\newpage
\section{Discrete Frequency Sampling Function}
\begin{lstlisting}[style=Matlab-editor,basicstyle=\footnotesize]
function delFn1 = sampledF(freq,fn)
    delFn1=zeros(1,length(freq));
    delFn1(find(mod(round(freq*1000),round(fn*1000))==0))=1;
    %impulses at f=fn,2fn,3fn... 
end
\end{lstlisting}

\newpage
\section{Continuous And Discrete Spectrum Component Calculation}
\begin{lstlisting}[style=Matlab-editor,basicstyle=\footnotesize]
%------------------------------------------------
%   getXcXd2()
%------------------------------------------------
%function [f,Xc,Xd]= getXcXd239(Tb,n,r,f_min,f_points,f_max)
    Xc=[];Xd=[];f=[];    
    % k --> No of codes words
    [k A] = getCodeMatrix(n,r);
    At=A';
    P_0=(1/k)*eye(k);
    P_k=(1/(k^2))*ones(k);
    
    R_0 = At*P_0*A;
    R_k = At*P_k*A;
    
    for f_indx = 0:f_points
      freq = f_indx*(f_max-f_min)/f_points + f_min;
      f =  [f freq];
      v=exp(j*2*pi*Tb*freq*(1:n));
      vt=v';
      Xc = [Xc v*(R_0-R_k)*vt];
      Xd = [Xd v*R_k*vt];
      
    end
end
\end{lstlisting}