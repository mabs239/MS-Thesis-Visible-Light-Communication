% Appendix A
% add following line for typesetting from subfiles
% !TeX root = ../uet_thesis.tex
\chapter{FPGA Implementation}
\label{AppendixB}
\lhead{Appendix B. \emph{FPGA Implementation}}

%\lstset{language=ASC}
\section{Encoder Module}
\label{sec:Encode2ts}
%\begin{lstlisting}[style=verilog-style,basicstyle=\footnotesize]
\begin{lstlisting}[style=verilog-style,basicstyle=\tiny]
//////////////////////////////////////////////////////////////////////////////////
// Module Name:	Encode2ts
// Project Name: 	VLC

// Engineer: 		Muhammad Abu Bakar Siddique
// email:			mabs239@gmail.com
// Create Date:  	09:52:41 11/25/2011 
// Design Name: 	Visible Light Communication
// Target Devices:Spartan-3 
// Tool versions: Nexys-2 from Digilent, Xilinx ISE 13.2
// Description: 	Encodes parallel data input to VR-MPPM encoded codeword 
//						Takes 'n' clock cycles for the conversion
//
// Dependencies: 	nCrROM.v
//
// Revision: 
// Revision 0.01 - File Created
// Additional Comments: 
//
//////////////////////////////////////////////////////////////////////////////////

module Encode2ts(m,frameSize,r,out,clk,start,complete,serOut);
	output [19:0] out;
	output complete;
	output serOut;

	input [19:0] m; //input number
	input [7:0] r; //pulsed slots
	input [7:0] frameSize; //slots per codeword n
	input clk;
	input start;
  
	reg [7:0] n; //index outRam
	reg [19:0] loopCount;
  	reg [19:0] outWord;	//Encoded output word, appears after 'n' clock periods
	reg [19:0] inWord;	//Encoded output word, appears after 'n' clock periods
	reg serialTemp;
	reg [7:0] unitCount;//Alternate to r
	reg overFlow;		//Output signal to indicate input value is greater than possible with given n and r  
	reg tempComplete;
	wire [19:0] y;		//getnCr module output connecte to this wire. Calculates nCr

	nCrROM ncr1(.nCr(y),.n(n),.r(unitCount));
	assign out	= outWord[19:0];
	assign complete = tempComplete;
	assign serOut = serialTemp;

	initial
	begin
		tempComplete<=0;
		loopCount <= 0;
		outWord<=0;
	end
	always @(posedge clk, posedge start)  
	begin		
		if(start)
		begin
			tempComplete<=0;
			loopCount <= frameSize;
			n <= frameSize-1;
			outWord<=0;
			inWord<=m;
			unitCount<=r;
		end
		else 
		begin
			if(loopCount!=0) //main loop
			begin
				tempComplete<=0;
				if(inWord>=y) 
				begin
					inWord<=inWord-y;
					outWord[n]<=1'b1;
					serialTemp<=1'b1;
					unitCount<=unitCount-1;
				end	
				else
				begin
					outWord[n]<=1'b0;
					serialTemp<=1'b0;
				end
				n<=n-1;
				loopCount<=loopCount-1;
			end
			else
			if (loopCount==0)
			begin
				tempComplete<=1;
			end
		end
	end
endmodule
//////////////////////////////////////////////////////////////////////////////////
\end{lstlisting}

\newpage
\section{Decoder Module}
\label{sec:Decode2ts}
\begin{lstlisting}[style=verilog-style,basicstyle=\tiny]
//////////////////////////////////////////////////////////////////////////////////
// Module Name:	Decode2ts
// Project Name: 	VLC

// Engineer: 		Muhammad Abu Bakar Siddique
// email:			mabs239@gmail.com
// Create Date:  	09:52:41 11/25/2011 
// Design Name: 	Visible Light Communication
// Target Devices:Spartan-3 
// Tool versions: Nexys-2 from Digilent, Xilinx ISE 13.2
// Description: 	Decodes input VR-MPPM codeword to origional symbol
//						
//
// Dependencies: 	nCrROM.v
//
// Revision: 
// Revision 0.01 - File Created
// Additional Comments: 
//
//////////////////////////////////////////////////////////////////////////////////

module Decode2ts(m,frameSize,r,out,clk,start,complete);
	input [19:0] m; //input number
	input [7:0] r; //pulsed slots
	input [7:0] frameSize; //slots per codeword n
	output [19:0] out;
	input clk;
	input start;
	output complete;

	reg [7:0] n; //index outRam
	reg [7:0] loopCount;
  	reg [19:0] outWord;	//Encoded output word, appears after 'n' clock periods
	reg [19:0] inWord;	//Encoded output word, appears after 'n' clock periods
	reg [7:0] unitCount;//Alternate to r
	reg tempComplete;
	reg overFlow;		//Output signal to indicate input value is greater than possible with given n and r  
	wire [19:0] y;		//getnCr module output connecte to this wire. Calculates nCr
	
	assign out	= outWord;
	assign complete = tempComplete;
	nCrROM ncr1(.nCr(y),.n(n),.r(unitCount));
	
	initial
	begin
		loopCount <= 0;
		outWord<=0;
		overFlow<=1'b0;
		tempComplete<=0;
	end

	always @(posedge clk, posedge start)
	begin
		if(start)
		begin
			loopCount<=frameSize;
			n <= frameSize-1;
			outWord<=0;
			inWord<=m;
			unitCount<=r;
			overFlow<=1'b0;
			tempComplete<=0;
		end
		else 
		begin
			if(loopCount!=0) //main loop
			begin
				tempComplete<=0;
				if(inWord[n]==1) 
				begin
					outWord<=outWord+y;
					unitCount<=unitCount-1;
				end	
				n<=n-1;
				loopCount<=loopCount-1;
				if (loopCount==1)
					tempComplete<=1;
			end
		end //~start
	end//always
endmodule
//////////////////////////////////////////////////////////////////////////////////
\end{lstlisting}


\newpage
\section{Clock Divider Module}
\label{sec:clk_div}
\begin{lstlisting}[style=verilog-style,basicstyle=\tiny]
//////////////////////////////////////////////////////////////////////////////////
// Module Name:	clk_div
// Project Name: 	VLC

// Engineer: 		Muhammad Abu Bakar Siddique
// email:			mabs239@gmail.com
// Create Date:  	09:52:41 11/25/2011 
// Design Name: 	Visible Light Communication
// Target Devices:Spartan-3 
// Tool versions: Nexys-2 from Digilent, Xilinx ISE 13.2
// Description: 	Divides clock to lower frequency, selectable through "newDiv" 
//						parameter
//
// Dependencies: 	
//
// Revision: 
// Revision 0.01 - File Created
// Additional Comments: 
//
//////////////////////////////////////////////////////////////////////////////////

module clk_div(clk,newClk,newDiv);
	//newDiv*2 cycles per new clock cycle
	input clk;
	input [25:0] newDiv;
	output reg newClk;
	reg [25:0] clkCount;
	initial begin clkCount <= 0; newClk<=0; end
	
	always @(posedge clk)
	begin
		clkCount <= clkCount + 1;
		
		//Increase newDiv to increase the delay: h2ffffff used to slow 
		// down nexys-2 board clock to low enough to be percievable by human eye
		if (clkCount >=newDiv)
		begin
			newClk=~newClk;
			clkCount<=0;
		end
	end
endmodule
//////////////////////////////////////////////////////////////////////////////////
\end{lstlisting}


\newpage

\section{Parallel-Serial Interface Module}
\label{sec:p2p}
\begin{lstlisting}[style=verilog-style,basicstyle=\tiny]
//////////////////////////////////////////////////////////////////////////////////
// Module Name:	p2p
// Project Name: 	VLC

// Engineer: 		Muhammad Abu Bakar Siddique
// email:			mabs239@gmail.com
// Create Date:  	09:52:41 11/25/2011 
// Design Name: 	Visible Light Communication
// Target Devices:Spartan-3 
// Tool versions: Nexys-2 from Digilent, Xilinx ISE 13.2
// Description: 	Parallel<->Serial interface. Sends and receives data over external 
//						serial link.
//
// Dependencies: 	
//
// Revision: 
// Revision 0.01 - File Created
// Additional Comments: 
//
//////////////////////////////////////////////////////////////////////////////////

module p2p(pin,pout,serIn,serOut,clk,start,frameSize,complete);
	input clk,start;
	input [19:0] pin;
	input serIn;
	input [7:0] frameSize;
	output [19:0] pout; 
	output serOut;
	output complete;
	
	reg tempComplete;
	reg [7:0] loopCount;	
	reg [7:0] n;
	reg [19:0] tempIn;
	reg [19:0] tempOut;
	
	assign complete = tempComplete;
	assign pout=tempOut;
	assign serOut = tempIn[n];
	
	initial
	begin
		loopCount <= 0;
		tempComplete<=0;
	end

	always @(negedge clk) if(!tempComplete) tempOut[n]<=serIn;
	
	always @(posedge clk, posedge start)  
	begin
		if(start)
		begin
			loopCount <= frameSize;
			n<=frameSize-1;
			tempComplete<=0;
			tempIn<=pin;
		end
		else 
		begin
			if(loopCount!=0) //main loop
			begin
				loopCount<=loopCount-1;
				n<=n-1;
			end
			
			else
			if(loopCount==0) //main loop
				tempComplete<=1'b1;
		end //not start
	end//always
endmodule

//////////////////////////////////////////////////////////////////////////////////
\end{lstlisting}

\newpage
\section{VR-MPPM encoded Parallel-Serial Interface Module}
\label{sec:ENp2p}
\begin{lstlisting}[style=verilog-style,basicstyle=\tiny]
//////////////////////////////////////////////////////////////////////////////////
// Module Name:	ENp2p
// Project Name: 	VLC

// Engineer: 		Muhammad Abu Bakar Siddique
// email:			mabs239@gmail.com
// Create Date:  	09:52:41 11/25/2011 
// Design Name: 	Visible Light Communication
// Target Devices:Spartan-3 
// Tool versions: Nexys-2 from Digilent, Xilinx ISE 13.2
// Description: 	Sends and receives VR-MPPM encoded data over VLC serial link.
//						
//
// Dependencies: 	clk_div.v, Encode2ts.v, p2p.v, Decode2ts.v
//
// Revision: 
// Revision 0.01 - File Created
// Additional Comments: 
//
//////////////////////////////////////////////////////////////////////////////////

module ENp2p(pin,pout,serIn,serOut,clk,start,frameSize,r,complete);
	input clk,start;
	input [19:0] pin;
	input serIn;
	input [7:0] frameSize;
	input [7:0] r;
	output [19:0] pout; 
	output serOut;
	output complete;
	parameter A=3'd0, B=3'd1, C=3'd2, D=3'd3, E=3'd4, F=3'd5, G=3'd6, H=3'd7;		
	
	reg tempClk;
	reg tComplete;
	reg [7:0] loopCount;	
	reg tStart;
	reg [2:0] state, nextState;
	reg enStart, pspStart, deStart;
	wire [19:0] enPsp, pspDe; //interconnect modules
	wire enComplete, pspComplete, deComplete;
	wire serLink;
	wire newClk;
	
	clk_div clkDiv1(.clk(clk),.newClk(newClk),.newDiv(50)); //50 --> 500kHz
	Encode2ts e1(.m(pin),.frameSize(frameSize),.r(r),.out(enPsp),
		.clk(clk),.start(enStart),.complete(enComplete));
	p2p pp1(.pin(enPsp),.pout(pspDe),.serIn(serIn),.serOut(serOut),
		.clk(newClk),.start(pspStart),.frameSize(frameSize),.complete(pspComplete));
	Decode2ts d1(.m(pspDe),.frameSize(frameSize),.r(r),.out(pout),
		.clk(clk),.start(deStart),.complete(deComplete));
	assign complete = tComplete;
		
	initial
	begin
		 enStart<=0;
		 deStart<=0;
		 pspStart<=0;
		 nextState<=A;
	end
	
	always @(posedge clk) state <= nextState;

	always @(posedge clk)
	begin
		case (state)
		A: if ((start==1)) nextState<=B; else nextState<=A;
		B: nextState <= C;
		C: if (enComplete) nextState<=D; else nextState<=C;
		D: nextState <= E;
		E: if (pspComplete) nextState<=F; else nextState<=E;
		F: nextState <= G;
		G: if (deComplete) nextState<=H; else nextState<=G;
		H: nextState<=A;
		endcase
	end
	
	//do
	always @(posedge clk)
	begin
		case (state)
		A: begin enStart<=0;pspStart <=0;deStart <=0;  end
		B: begin tComplete<=0;enStart<=1;  end
		C:enStart<=0;
		D:pspStart <=1;
		E:pspStart <=0;
		F:deStart <=1;
		G: begin deStart <=0;end
		H: tComplete<=1; 
		endcase
	end
endmodule
//////////////////////////////////////////////////////////////////////////////////
\end{lstlisting}

\newpage
\section{Module for Scanning VR-MPPM Codes}
\label{sec:scanCodes}
\begin{lstlisting}[style=verilog-style,basicstyle=\tiny]
//////////////////////////////////////////////////////////////////////////////////
// Module Name:	ScanCodes
// Project Name: 	VLC

// Engineer: 		Muhammad Abu Bakar Siddique
// email:			mabs239@gmail.com
// Create Date:  	09:52:41 11/25/2011 
// Design Name: 	Visible Light Communication
// Target Devices:Spartan-3 
// Tool versions: Nexys-2 from Digilent, Xilinx ISE 13.2
// Description: 	Symbol error rate evaluation on VR-MPPM encoded visible light	
//						communication channel.
//
// Dependencies: 	nCrROM.v, ENp2p.v
//
// Revision: 
// Revision 0.01 - File Created
// Additional Comments: 
//
//////////////////////////////////////////////////////////////////////////////////

module scanCodes(frameSize,r,serIn,serOut,clk,start,complete,errorCount,dout,reset);
	input [7:0] frameSize;
	input [7:0] r;
	input serIn;
	input reset;
	output serOut;
	output complete;
	output [15:0] errorCount;
	input clk,start;
	output [15:0] dout;

	parameter A=3'd0, B=3'd1, C=3'd2, D=3'd3, E=3'd4, F=3'd5, G=3'd6, H=3'd7;
	parameter numSymbols=16'hEFFF;
	
	reg tStart;
	reg [15:0] pin; //data to be send serially
	reg [15:0] tErrorCount;
	reg [15:0] symbolCount;
	reg [3:0] state, nextState;
	reg scanComplete;
	wire enComplete;
	wire [15:0] y;
	wire [15:0] pout; //data received in serially

	assign dout=symbolCount;	
	assign errorCount=tErrorCount;	
	assign complete=scanComplete;	
	nCrROM ncr1(.nCr(y),.n(frameSize),.r(r));
	ENp2p enp2p1(.pin(pin),.pout(pout),.serIn(serIn),
		.serOut(serOut),.clk(clk),.start(tStart),
		.frameSize(frameSize+1'b1),.r(r),.complete(enComplete));
	initial
	begin
		pin<=16'd0;
		symbolCount<=16'h0000;
		tStart<=0;
		nextState<=0;
		state<=0;
		scanComplete<=0;
		tErrorCount<=0;
	end

	always @(negedge clk) state <= nextState;
	//nextState
	always @(clk or state)
	begin
		case (state)
		0: if (start) nextState<=1; else nextState<=0;
		1: nextState <= 2; //start=1
		2: nextState <= 3; //start==0
		3: nextState <= 4; //start==0
		4: nextState <= 5; //start==0
		5: if (enComplete) nextState<=6; else nextState<=5; //transmitting (completed?)
		6: nextState <= 7;
		7: nextState <= 8;
		8: nextState <= 9;
		9: nextState <= 10;
		10: nextState <= 11;
		11: nextState <= 12;
		12: if (scanComplete) nextState<=0; else nextState<=13;
		13: nextState <= 14;
		14: nextState <= 15;
		15: nextState <= 1;
		endcase
	end
	//do
	always @(posedge clk)
	begin
		case (state)
		0: begin 
				tStart <=0;
				scanComplete<=0;
				pin<=0;
			end
		1: begin tStart<=1; end // <= has problems in vvp simulation
		2: begin tStart<=0; end
		3: begin tStart<=0; end //wait
		9: begin
				if(pin!=pout) tErrorCount<=tErrorCount+1'b1;
			end
		10:if(pin>=(y-1)) pin<=0; else pin<=pin+1; 
		11: begin
				if(symbolCount>=(numSymbols)) begin
					symbolCount<=0; 
					scanComplete<=1;
				end
				else
					symbolCount<=symbolCount+1'b1; 
			end
		endcase
	end
endmodule
//////////////////////////////////////////////////////////////////////////////////
\end{lstlisting}


\newpage
\section{Seven Segment Display Driver Module}
\label{sec:LED_7seg}
\begin{lstlisting}[style=verilog-style,basicstyle=\tiny]
////////////////////////////////////////////////////////////////////////////////
//
// Create Date:    10/05/07
// Module Name:    LED_7seg
// Description:    Convert a 16 bit number into Hex output on the 7seg LED
//
// Revision:
// Revision 0.01 - File Created
//
// File format:		This file has been formated to use tabstop of 4
//
// Development platform: Spartan-3 Nexys from Digilent
//
// Copyright (C) 2007, Rick Huang
//   
// This library is free software; you can redistribute it and/or
// modify it under the terms of the GNU Lesser General Public
// License as published by the Free Software Foundation; either
// version 2.1 of the License, or (at your option) any later version.
// 
// This library is distributed in the hope that it will be useful,
// but WITHOUT ANY WARRANTY; without even the implied warranty of
// MERCHANTABILITY or FITNESS FOR A PARTICULAR PURPOSE.  See the GNU
// Lesser General Public License for more details.
// 
// You should have received a copy of the GNU Lesser General Public
// License along with this library; if not, write to the Free Software
// Foundation, Inc., 51 Franklin Street, Fifth Floor, Boston, MA  02110-1301  USA
// 
////////////////////////////////////////////////////////////////////////////////

module LED_7seg (mclk, reset, an, ssg, num, pt);
	input mclk;
	input reset;
	output [3:0] an;			// Anode connection
	output [7:0] ssg;			// Cathod connection
	input [15:0] num;			// Number input
	input [3:0] pt;				// Decimal point

	// Divide the incoming clock to a lower frequency
	reg [15:0] clk_divider;
	reg clk_low;
	always @ (posedge mclk)
	begin
		if(reset)
			clk_divider <= 0;
		else begin
			clk_divider <= clk_divider + 1;
			if(clk_divider == 0)
				clk_low <= 1;
			else
				clk_low <= 0;
		end
	end

	// Scan the 7-seg LED
	reg [1:0] digit_scan;
	reg [3:0] an;
	reg [3:0] digit_num;
	reg dec_pt;
	
	// Simple ROM to convert digit to LED segments
	assign ssg[6:0] = (digit_num == 4'h0) ? 7'b1000000 :
					  (digit_num == 4'h1) ? 7'b1111001 :
					  (digit_num == 4'h2) ? 7'b0100100 :
					  (digit_num == 4'h3) ? 7'b0110000 :
					  (digit_num == 4'h4) ? 7'b0011001 :
					  (digit_num == 4'h5) ? 7'b0010010 :
					  (digit_num == 4'h6) ? 7'b0000010 :
					  (digit_num == 4'h7) ? 7'b1111000 :
					  (digit_num == 4'h8) ? 7'b0000000 :
					  (digit_num == 4'h9) ? 7'b0011000 :
					  (digit_num == 4'ha) ? 7'b0001000 :
					  (digit_num == 4'hb) ? 7'b0000011 :
					  (digit_num == 4'hc) ? 7'b1000110 :
					  (digit_num == 4'hd) ? 7'b0100001 :
					  (digit_num == 4'he) ? 7'b0000110 :
					  (digit_num == 4'hf) ? 7'b0001110 : 7'b1111111;
	assign ssg[7] = !dec_pt;
	always @ (posedge mclk)
	begin
		if(reset)
			digit_scan <= 0;
		else if(clk_low)
		begin
			digit_scan <= digit_scan + 1;
			case (digit_scan)
				2'b00:
				begin
					an <= 4'b1110;
					digit_num <= num[3:0];
					dec_pt <= pt[0] ? 1 : 0;
				end
				2'b01:
				begin
					an <= 4'b1101;
					digit_num <= num[7:4];
					dec_pt <= pt[1] ? 1 : 0;
				end
				2'b10:
				begin
					an <= 4'b1011;
					digit_num <= num[11:8];
					dec_pt <= pt[2] ? 1 : 0;
				end
				2'b11:
				begin
					an <= 4'b0111;
					digit_num <= num[15:12];
					dec_pt <= pt[3] ? 1 : 0;
				end
			endcase
		end
	end
endmodule
//////////////////////////////////////////////////////////////////////////////////
\end{lstlisting}


\newpage
\section{Toplevel Implementation Module}
\label{sec:vlc}
\begin{lstlisting}[style=verilog-style,basicstyle=\tiny]
`timescale 1ns / 1ps
//////////////////////////////////////////////////////////////////////////////////
// Module Name:	VLC
// Project Name: 	VLC

// Engineer: 		Muhammad Abu Bakar Siddique
// email:		mabs239@gmail.com
// Create Date:  	09:52:41 11/25/2011 
// Design Name: 	Visible Light Communication
// Target Devices:Spartan-3 
// Tool versions: Nexys-2 from Digilent, Xilinx ISE 13.2
// Description: 	Symbol error rate evaluation on VR-MPPM encoded visible light	
//			communication channel
//
// Dependencies: 	ScanCodes.v, toplevel.ucf
//
// Revision: 
// Revision 0.01 - File Created
// Additional Comments: 
//
//////////////////////////////////////////////////////////////////////////////////

module VLC(switches,BTN0,BTN1,BTN2,BTN3,led,JA1,JA2,JA3,clk,anodes,sevenseg);
	input [7:0] switches;
	input BTN3; //encoder start
	input BTN2; //pout or errors
	input BTN1; //
	input BTN0; //reset
	input JA1;
	input clk;
	output [7:0] led;
	output JA2;
	output JA3;
	output [3:0] anodes;
	output [7:0] sevenseg;
	wire newClk;
	wire [16:0] seg7;
	wire [15:0] errorCount;
	wire [15:0] dout;
	wire serOut;

	assign seg7=(BTN2)?errorCount:dout;
	assign JA2=~serOut;
	assign JA3=serOut;
	assign led[7] = newClk;	
	assign led[6] = JA2;	
	assign led[5] = JA1;	
	assign led[4:0]=dout;

	scanCodes m7(.frameSize(switches[7:4]),.r(switches[3:0]),.serIn(JA1),
		.serOut(serOut),.clk(clk),.start(BTN3),
		.errorCount(errorCount),.dout(dout),.reset(BTN0));
	LED_7seg m5(.mclk(clk),.reset(BTN0),.an(anodes),.ssg(sevenseg),.num(seg7));	
endmodule

//////////////////////////////////////////////////////////////////////////////////
\end{lstlisting}


\newpage
\section{Module to Evaluate Binomial Coefficients }
\label{sec:nCrROM}
\begin{lstlisting}[style=verilog-style,basicstyle=\tiny]
//////////////////////////////////////////////////////////////////////////////////
// Module Name:	nCrROM
// Project Name: 	VLC

// Engineer: 		Muhammad Abu Bakar Siddique
// email:			mabs239@gmail.com
// Create Date:  	09:52:41 11/25/2011 
// Design Name: 	Visible Light Communication
// Target Devices:Spartan-3 
// Tool versions: Nexys-2 from Digilent, Xilinx ISE 13.2
// Description: 	Evaluates nCr, combinations of n things taken r at a time
//						
//
// Dependencies: 	
//
// Revision: 
// Revision 0.01 - File Created
// Additional Comments: 
//
//////////////////////////////////////////////////////////////////////////////////

module nCrROM(nCr,n,r);
	output [15:0] nCr;  //n*r size
	input [3:0] n,r; //frame size 1~16
	reg [15:0] romTable[0:255];
	assign nCr = romTable[{n,r}];
	
	initial
	begin //rom table
		romTable[0]  <=  16'd1;
		romTable[1]  <=  16'd0;
		romTable[2]  <=  16'd0;
		romTable[3]  <=  16'd0;
		romTable[4]  <=  16'd0;
		romTable[5]  <=  16'd0;
		romTable[6]  <=  16'd0;
		romTable[7]  <=  16'd0;
		romTable[8]  <=  16'd0;
		romTable[9]  <=  16'd0;
		romTable[10]  <=  16'd0;
		romTable[11]  <=  16'd0;
		romTable[12]  <=  16'd0;
		romTable[13]  <=  16'd0;
		romTable[14]  <=  16'd0;
		romTable[15]  <=  16'd0;
		romTable[16]  <=  16'd1;
		romTable[17]  <=  16'd1;
		romTable[18]  <=  16'd0;
		romTable[19]  <=  16'd0;
		romTable[20]  <=  16'd0;
		romTable[21]  <=  16'd0;
		romTable[22]  <=  16'd0;
		romTable[23]  <=  16'd0;
		romTable[24]  <=  16'd0;
		romTable[25]  <=  16'd0;
		romTable[26]  <=  16'd0;
		romTable[27]  <=  16'd0;
		romTable[28]  <=  16'd0;
		romTable[29]  <=  16'd0;
		romTable[30]  <=  16'd0;
		romTable[31]  <=  16'd0;
		romTable[32]  <=  16'd1;
		romTable[33]  <=  16'd2;
		romTable[34]  <=  16'd1;
		romTable[35]  <=  16'd0;
		romTable[36]  <=  16'd0;
		romTable[37]  <=  16'd0;
		romTable[38]  <=  16'd0;
		romTable[39]  <=  16'd0;
		romTable[40]  <=  16'd0;
		romTable[41]  <=  16'd0;
		romTable[42]  <=  16'd0;
		romTable[43]  <=  16'd0;
		romTable[44]  <=  16'd0;
		romTable[45]  <=  16'd0;
		romTable[46]  <=  16'd0;
		romTable[47]  <=  16'd0;
		romTable[48]  <=  16'd1;
		romTable[49]  <=  16'd3;
		romTable[50]  <=  16'd3;
		romTable[51]  <=  16'd1;
		romTable[52]  <=  16'd0;
		romTable[53]  <=  16'd0;
		romTable[54]  <=  16'd0;
		romTable[55]  <=  16'd0;
		romTable[56]  <=  16'd0;
		romTable[57]  <=  16'd0;
		romTable[58]  <=  16'd0;
		romTable[59]  <=  16'd0;
		romTable[60]  <=  16'd0;
		romTable[61]  <=  16'd0;
		romTable[62]  <=  16'd0;
		romTable[63]  <=  16'd0;
		romTable[64]  <=  16'd1;
		romTable[65]  <=  16'd4;
		romTable[66]  <=  16'd6;
		romTable[67]  <=  16'd4;
		romTable[68]  <=  16'd1;
		romTable[69]  <=  16'd0;
		romTable[70]  <=  16'd0;
		romTable[71]  <=  16'd0;
		romTable[72]  <=  16'd0;
		romTable[73]  <=  16'd0;
		romTable[74]  <=  16'd0;
		romTable[75]  <=  16'd0;
		romTable[76]  <=  16'd0;
		romTable[77]  <=  16'd0;
		romTable[78]  <=  16'd0;
		romTable[79]  <=  16'd0;
		romTable[80]  <=  16'd1;
		romTable[81]  <=  16'd5;
		romTable[82]  <=  16'd10;
		romTable[83]  <=  16'd10;
		romTable[84]  <=  16'd5;
		romTable[85]  <=  16'd1;
		romTable[86]  <=  16'd0;
		romTable[87]  <=  16'd0;
		romTable[88]  <=  16'd0;
		romTable[89]  <=  16'd0;
		romTable[90]  <=  16'd0;
		romTable[91]  <=  16'd0;
		romTable[92]  <=  16'd0;
		romTable[93]  <=  16'd0;
		romTable[94]  <=  16'd0;
		romTable[95]  <=  16'd0;
		romTable[96]  <=  16'd1;
		romTable[97]  <=  16'd6;
		romTable[98]  <=  16'd15;
		romTable[99]  <=  16'd20;
		romTable[100]  <=  16'd15;
		romTable[101]  <=  16'd6;
		romTable[102]  <=  16'd1;
		romTable[103]  <=  16'd0;
		romTable[104]  <=  16'd0;
		romTable[105]  <=  16'd0;
		romTable[106]  <=  16'd0;
		romTable[107]  <=  16'd0;
		romTable[108]  <=  16'd0;
		romTable[109]  <=  16'd0;
		romTable[110]  <=  16'd0;
		romTable[111]  <=  16'd0;
		romTable[112]  <=  16'd1;
		romTable[113]  <=  16'd7;
		romTable[114]  <=  16'd21;
		romTable[115]  <=  16'd35;
		romTable[116]  <=  16'd35;
		romTable[117]  <=  16'd21;
		romTable[118]  <=  16'd7;
		romTable[119]  <=  16'd1;
		romTable[120]  <=  16'd0;
		romTable[121]  <=  16'd0;
		romTable[122]  <=  16'd0;
		romTable[123]  <=  16'd0;
		romTable[124]  <=  16'd0;
		romTable[125]  <=  16'd0;
		romTable[126]  <=  16'd0;
		romTable[127]  <=  16'd0;
		romTable[128]  <=  16'd1;
		romTable[129]  <=  16'd8;
		romTable[130]  <=  16'd28;
		romTable[131]  <=  16'd56;
		romTable[132]  <=  16'd70;
		romTable[133]  <=  16'd56;
		romTable[134]  <=  16'd28;
		romTable[135]  <=  16'd8;
		romTable[136]  <=  16'd1;
		romTable[137]  <=  16'd0;
		romTable[138]  <=  16'd0;
		romTable[139]  <=  16'd0;
		romTable[140]  <=  16'd0;
		romTable[141]  <=  16'd0;
		romTable[142]  <=  16'd0;
		romTable[143]  <=  16'd0;
		romTable[144]  <=  16'd1;
		romTable[145]  <=  16'd9;
		romTable[146]  <=  16'd36;
		romTable[147]  <=  16'd84;
		romTable[148]  <=  16'd126;
		romTable[149]  <=  16'd126;
		romTable[150]  <=  16'd84;
		romTable[151]  <=  16'd36;
		romTable[152]  <=  16'd9;
		romTable[153]  <=  16'd1;
		romTable[154]  <=  16'd0;
		romTable[155]  <=  16'd0;
		romTable[156]  <=  16'd0;
		romTable[157]  <=  16'd0;
		romTable[158]  <=  16'd0;
		romTable[159]  <=  16'd0;
		romTable[160]  <=  16'd1;
		romTable[161]  <=  16'd10;
		romTable[162]  <=  16'd45;
		romTable[163]  <=  16'd120;
		romTable[164]  <=  16'd210;
		romTable[165]  <=  16'd252;
		romTable[166]  <=  16'd210;
		romTable[167]  <=  16'd120;
		romTable[168]  <=  16'd45;
		romTable[169]  <=  16'd10;
		romTable[170]  <=  16'd1;
		romTable[171]  <=  16'd0;
		romTable[172]  <=  16'd0;
		romTable[173]  <=  16'd0;
		romTable[174]  <=  16'd0;
		romTable[175]  <=  16'd0;
		romTable[176]  <=  16'd1;
		romTable[177]  <=  16'd11;
		romTable[178]  <=  16'd55;
		romTable[179]  <=  16'd165;
		romTable[180]  <=  16'd330;
		romTable[181]  <=  16'd462;
		romTable[182]  <=  16'd462;
		romTable[183]  <=  16'd330;
		romTable[184]  <=  16'd165;
		romTable[185]  <=  16'd55;
		romTable[186]  <=  16'd11;
		romTable[187]  <=  16'd1;
		romTable[188]  <=  16'd0;
		romTable[189]  <=  16'd0;
		romTable[190]  <=  16'd0;
		romTable[191]  <=  16'd0;
		romTable[192]  <=  16'd1;
		romTable[193]  <=  16'd12;
		romTable[194]  <=  16'd66;
		romTable[195]  <=  16'd220;
		romTable[196]  <=  16'd495;
		romTable[197]  <=  16'd792;
		romTable[198]  <=  16'd924;
		romTable[199]  <=  16'd792;
		romTable[200]  <=  16'd495;
		romTable[201]  <=  16'd220;
		romTable[202]  <=  16'd66;
		romTable[203]  <=  16'd12;
		romTable[204]  <=  16'd1;
		romTable[205]  <=  16'd0;
		romTable[206]  <=  16'd0;
		romTable[207]  <=  16'd0;
		romTable[208]  <=  16'd1;
		romTable[209]  <=  16'd13;
		romTable[210]  <=  16'd78;
		romTable[211]  <=  16'd286;
		romTable[212]  <=  16'd715;
		romTable[213]  <=  16'd1287;
		romTable[214]  <=  16'd1716;
		romTable[215]  <=  16'd1716;
		romTable[216]  <=  16'd1287;
		romTable[217]  <=  16'd715;
		romTable[218]  <=  16'd286;
		romTable[219]  <=  16'd78;
		romTable[220]  <=  16'd13;
		romTable[221]  <=  16'd1;
		romTable[222]  <=  16'd0;
		romTable[223]  <=  16'd0;
		romTable[224]  <=  16'd1;
		romTable[225]  <=  16'd14;
		romTable[226]  <=  16'd91;
		romTable[227]  <=  16'd364;
		romTable[228]  <=  16'd1001;
		romTable[229]  <=  16'd2002;
		romTable[230]  <=  16'd3003;
		romTable[231]  <=  16'd3432;
		romTable[232]  <=  16'd3003;
		romTable[233]  <=  16'd2002;
		romTable[234]  <=  16'd1001;
		romTable[235]  <=  16'd364;
		romTable[236]  <=  16'd91;
		romTable[237]  <=  16'd14;
		romTable[238]  <=  16'd1;
		romTable[239]  <=  16'd0;
		romTable[240]  <=  16'd1;
		romTable[241]  <=  16'd15;
		romTable[242]  <=  16'd105;
		romTable[243]  <=  16'd455;
		romTable[244]  <=  16'd1365;
		romTable[245]  <=  16'd3003;
		romTable[246]  <=  16'd5005;
		romTable[247]  <=  16'd6435;
		romTable[248]  <=  16'd6435;
		romTable[249]  <=  16'd5005;
		romTable[250]  <=  16'd3003;
		romTable[251]  <=  16'd1365;
		romTable[252]  <=  16'd455;
		romTable[253]  <=  16'd105;
		romTable[254]  <=  16'd15;
		romTable[255]  <=  16'd1;
	end
endmodule
//////////////////////////////////////////////////////////////////////////////////
\end{lstlisting}


\newpage
\section{User Constraint File}
\label{sec:toplevel}
\begin{lstlisting}[style=verilog-style,basicstyle=\tiny]
#Nexys-2 Digilent
NET "clk"   LOC = "B8";

NET "BTN3" LOC = "H13";
NET "BTN3" CLOCK_DEDICATED_ROUTE = FALSE;
NET "BTN2" LOC = "E18";
NET "BTN2" CLOCK_DEDICATED_ROUTE = FALSE;
NET "BTN1" LOC = "D18";
NET "BTN1" CLOCK_DEDICATED_ROUTE = FALSE;
NET "BTN0" LOC = "B18";
NET "BTN0" CLOCK_DEDICATED_ROUTE = FALSE;

NET "JA1" LOC = "L15"; #Serial input
NET "JA2" LOC = "K12"; #Serial output
NET "JA3" LOC = "L17"; #Inverted serial output


NET "led<0>" LOC = "J14";
NET "led<1>" LOC = "J15";
NET "led<2>" LOC = "K15";
NET "led<3>" LOC = "K14";
NET "led<4>" LOC = "E17";
NET "led<5>" LOC = "P15";
NET "led<6>" LOC = "F4";
NET "led<7>" LOC = "R4";

NET "anodes<0>"  LOC = "F17"; 
NET "anodes<1>"  LOC = "H17"; 
NET "anodes<2>"  LOC = "C18"; 
NET "anodes<3>"  LOC = "F15"; 
 
NET "sevenseg<0>" LOC = "L18"; 
NET "sevenseg<1>" LOC = "F18"; 
NET "sevenseg<2>" LOC = "D17"; 
NET "sevenseg<3>" LOC = "D16"; 
NET "sevenseg<4>" LOC = "G14"; 
NET "sevenseg<5>" LOC = "J17"; 
NET "sevenseg<6>" LOC = "H14"; 

NET "switches<0>" LOC = "G18"; 
NET "switches<1>" LOC = "H18"; 
NET "switches<2>" LOC = "K18";
NET "switches<3>" LOC = "K17";
NET "switches<4>" LOC = "L14"; 
NET "switches<5>" LOC = "L13"; 
NET "switches<6>" LOC = "N17";
NET "switches<7>" LOC = "R17";
//////////////////////////////////////////////////////////////////////////////////
\end{lstlisting}
